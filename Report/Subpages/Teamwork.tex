\section{Teamwork}
In this section the teamwork is described. The requirements were 3-4 people in a group and no more than two persons should share a computer while programming. Another requirement was the use of \emph{Continuous Integration}, which helps the developers commit to a shared build server multiple times during the development process.

\subsection{Strategies}
Earlier experience from working with some of the strategies from \emph{Extreme Programming} in the course I4SWD, this was found suitable for the software development of the Air Traffic Monitor too. \\
One of these strategies was \emph{Pair programming}. Code is then written by pairs which shares the workstation. One will be in control of the keyboard and write the code while the other will watch the code and work towards the best implementation. The pair switches place every now and then. This ensures that both programmers are engaged in the software.

\subsection{Continuous integration}
As the group was divided into two pairs of developers each working on classes of their own, the continuous integration helped the two groups to gain a shared understanding on how the software development progress.\\ 
Another benefit was the automatic generated code coverage report and software quality metrics which were used to determine whether the written software and tests were satisfying. If not, it was easy to gather information where the code standards should be optimized for better statistics.\\ 
As with every other git project a version history was obtained making it simple to revert to a previous build if changes caused a broken build.\\